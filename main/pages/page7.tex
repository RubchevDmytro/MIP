\section*{VV-B: Ciele, zámery a výstupy projektu / Project objectives, aims and outputs}
\subsection*{VV-B-01: Kľúčové slová / Key words}

AI, drон, poľnohospodárstvo, presné poľnohospodárstvo, multispektrálne snímanie, edge computing, TRL 4–6, Green Deal, Copernicus, udržateľnosť  \\
AI, drone, agriculture, precision farming, multispectral imaging, edge computing, TRL 4–6, Green Deal, Copernicus, sustainability
\subsection*{VV-B-02: Ciele projektu / Project objectives}
Hlavným cieľom projektu je vývoj inteligentného dronového systému na monitorovanie poľnohospodárskych plodín s využitím AI na detekciu chorôb, škodcov, stresu a výnosov v reálnom čase. Systém integruje multispektrálne snímanie (DJI P4 Multispectral), AI modely (YOLOv8, U-Net) a edge computing (NVIDIA Jetson) na dron pre okamžité rozhodovanie bez závislosti od cloudu. Ďalšie ciele:  \\
1) Zvýšiť presnosť detekcie o ≥20 \% oproti existujúcim riešeniam pomocou fúzie dát z Copernicus a lokálnych senzorov.  \\
2) Dosiahnuť TRL 6 (prototyp testovaný v reálnom prostredí na 5 ha v regióne Bratislava).  \\
3) Vyvinúť softvérový prototyp s používateľským rozhraním pre farmárov (mobilná aplikácia). \\ 
4) Znížiť spotrebu pesticídov o 15–25 \% a podporiť ciele Zelenej dohody pre Európu (EU Green Deal).  \\
5) Vzdelávať 4 mladých výskumníkov (PhD študenti) v oblasti AI a dronov.  \\

The main objective is to develop an intelligent drone system for real-time crop monitoring using AI to detect diseases, pests, stress, and yields. The system integrates multispectral imaging (DJI P4 Multispectral), AI models (YOLOv8, U-Net), and edge computing (NVIDIA Jetson) for on-drone decision-making. Additional objectives:  
1) Increase detection accuracy by ≥20 \% using Copernicus + local sensor fusion.  \\
2) Achieve TRL 6 (prototype tested on 5 ha in Bratislava region).  \\
3) Develop a software prototype with farmer UI (mobile app).  \\
4) Reduce pesticide use by 15–25 \% and support EU Green Deal goals.  \\
5) Train 4 young researchers (PhD students) in AI and drones.
\subsection*{VV-B-03: Uveďte zaradenie projektu do príslušnej TRL stupnice / Indicate the Project's TRL level}
TRL 4–6
\subsection*{VV-B-03: Uveďte zdôvodnenie zaradenia projektu do príslušnej TRL stupnice/ Explain the indicated Project's TRL level}
Začiatok: TRL 4 – overenie AI modelov (YOLOv8) na simulovaných a historických dátach v laboratóriu STU.  
Koniec: TRL 6 – plne funkčný prototyp (dron s Jetson + AI) testovaný na 5 ha poľnohospodárskej plochy v regióne Bratislava s reálnymi farmármi.  

Start: TRL 4 – validation of AI models (YOLOv8) on simulated/historical data in STU lab.  
End: TRL 6 – fully functional prototype (drone with Jetson + AI) tested on 5 ha farmland in Bratislava region with real farmers.\\ \\
\subsection*{VV-B-03: Uveďte zaradenie výstupov projektu do prislušnej TRL stupnice s popisom charakteru výstupov / Indicate the inclusion of project outputs in the relevant TRL scale with the description of the outputs}
Výstupy:  
1) Softvérový prototyp AI modelov – TRL 6 (nasadenie na dron, reálne dáta).  
2) Integrácia multispektrálnych dát + Copernicus  – TRL 5 (test v kontrolovanom prostredí).  
3) Mobilná aplikácia pre farmárov – TRL 6 (použitie v praxi).  
4) Metodika optimalizácie pesticídov – TRL 6 (overená redukcia 15–25 \%).  

Outputs:  
1) AI software prototype – TRL 6 (deployed on drone, real data).  
2) Multispectral + Copernicus data fusion  – TRL 5 (tested in controlled environment).  
3) Farmer mobile app – TRL 6 (used in practice).  
4) Pesticide optimization methodology – TRL 6 (verified 15–25 \% reduction).
\subsection*{VV-B-04: Využitie výsledkov riešenia v praxi - Odberateľ (realizátor) výsledkov je žiadateľ / Project outcomes applications in practise - Outcomes customes (user) is applicant }
Odberateľ: Slovenskí farmári (napr. PD Horné Orechové, PD Senec) – priame použitie na presné postreky a plánovanie zavlažovania.  \\
Customer: Slovak farmers (e.g., PD Horné Orechové, PD Senec) – direct use for precision spraying and irrigation planning.
\subsection*{VV-B-04: Využitie výsledkov riešenia v praxi - Odberateľ (realizátor) výsledkov je iný odberateľ / Project outcomes applications in practise - Outcomes customes (user) is other user}
Iný odberateľ: Ministerstvo pôdohospodárstva SR, Európska komisia (Copernicus) – podpora politík udržateľného poľnohospodárstva.\\  
Other user: Ministry of Agriculture of SR, European Commission (Copernicus) – support for sustainable farming policies.
\subsection*{VV-B-04: Využitie výsledkov riešenia v praxi - Iný odberateľ (názov organizácie) / Project outcomes applications in practise - Other user (name of the organization) }
Názov: Agroinštitút Nitra, štátný podnik  \\
Name: Agroinštitút Nitra, state enterprise
\subsection*{VV-B-05: Prehľad plánovaných výstupov a prínosov projektu v nadväznosti na etapy riešenia projektu, ktoré budú verejne dostupné / An overview of the planned outputs and contributions of the project with regard to the stages of the project implementation, which will be publicly available}
Etapa 1 (2026):  
- Dataset (500+ multispektrálnych snímok) – open-source na GitHub.  
- Publikácia v časopise (MDPI Sensors).  \\
Etapa 2 (2027):  
- AI model (YOLOv8 fine-tuned) – open-source (Hugging Face). 
- Workshop pre farmárov (20 účastníkov). \\ 
Etapa 3 (2028):  
- Prototyp dronu + app – demo video na YouTube.  
- Finálna metodika (PDF) – web STU.  
- 2x konferenčná publikácia (AGRITECHNICA).  
\\ 
Stage 1 (2026):  
- Dataset (500+ multispectral images) – open-source on GitHub.  
- Journal paper (MDPI Sensors).  
Stage 2 (2027):  
- AI model (YOLOv8 fine-tuned) – open-source (Hugging Face).  
- Farmer workshop (20 participants).  

Stage 3 (2028):  
- Drone prototype + app – demo video on YouTube.  
- Final methodology (PDF) – STU website.  
- 2x conference papers (AGRITECHNICA).

