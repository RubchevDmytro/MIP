%SLOVAK
\section*{VV-F – VECNÝ ZÁMER PROJEKTU}

\begin{center}
\textbf{\textcolor{blue}{Záväzná osnova pre aplikovaný výskum a vývoj}}
\end{center}

\raggedright

\begin{tabular}{|p{3cm}|p{10cm}|}
\hline       
\\Názov projektu & AGRI-DRONE AI \\ \hline
\\Zodpovedný riešiteľ  & Martin Sabo \\ \hline
\\Žiadateľ & Rubchev Dmytro \\ \hline
\\Štatutárny/i zástupca/ovia žiadateľa & Ján Lang \\ \hline
\bottomrule
\end{tabular}

\section{Excelentnosť}

Projekt „Vývoj inteligentného dronového systému na poľnohospodárske monitorovanie s AI analýzou (AGRI-DRONE-AI)“ si kladie za cieľ vytvoriť integrovanú platformu, ktorá kombinuje multispektrálne snímanie dronmi, spracovanie na okraji AI a fúziu údajov zo satelitného počasia na zlepšenie presného poľnohospodárstva v Slovenskej republike.

\subsection*{Originalita spočíva v:}
\begin{enumerate}[label=(\alph*)]
    \item Inference AI v reálnom čase na palube drona pre detekciu škodcov, chorôb a stresu spôsobeného nedostatkom živín priamo v jednotke okrajového výpočtu drona;
    \item Prediktívna analytika, ktorá spája snímky z dronov s dátami zo satelitov Copernicus EU a miestnych meteorologických staníc;
    \item Akčné agronomické palubné dosky poskytujúce odporúčania pre farmárov na úrovni polí do niekoľkých minút od letu.
\end{enumerate}

Súčasné riešenia pre poľnohospodárske drony (napr. DJI Agriculture, PrecisionHawk) väčšinou poskytujú surové snímky, ktoré si vyžadujú offline spracovanie; náš koncept skracuje cyklus rozhodovania a znižuje závislosť od externých IT infraštruktúr.

\subsection*{Stav techniky}
Očakáva sa, že trh s poľnohospodárskymi dronmi v EÚ dosiahne do roku 2025 hodnotu \euro 7.46 miliardy EUR.. Výskumy ukazujú, že včasná detekcia škodcov môže znížiť stratu výnosov o 15-25 \% a znížiť spotrebu pesticídov až o 30 \%. Tím vychádza z predchádzajúcich národných pilotných štúdií v multispektrálnom monitorovaní pšenice a kukurice (2023-24) a interného prototypu na rozpoznávanie chorôb listov na základe CNN (presnosť $\approx 92\,\%$)
.
\subsection*{Ciele projektu:}
\begin{itemize}
    \item Navrhnúť architektúru senzorov dronov a okrajovú AI pipeline na hodnotenie zdravia plodín.
    \item Vyvinúť a trénovať modely počítačového videnia na rozpoznávanie škodcov/chorôb a stresu spôsobeného zavlažovaním.
    \item Integrovať prediktívne modely s dátami z meteorologických/poľnohospodárskych senzorov pre pokyny na zlepšenie výnosu.
    \item Overiť prototyp na minimálne 200 ha pilotných fariem v západnom a východnom Slovensku.
    \item Dodať overený softvérový prototyp, metodiku integrácie AI a smernice pre implementáciu najlepších praktík.
\end{itemize}

\subsection*{Fezibilita}
Dostupnosť komerčných dronov a GPU, robí tieto ciele realistické v priebehu 36 mesiacov.

\subsection*{Kľúčové výstupy hlavného výskumníka (posledných 5 rokov – ilustratívne):}
\begin{itemize}
    \item Nie
\end{itemize}

\subsection*{Kompetencie partnerov:}
\begin{itemize}
    \item Nie
\end{itemize}

\section{Dopad}

Očakáva sa, že systém zvýši výnosy plodín o 15-25 \% a zníži spotrebu pesticídov o 20-30 \%, čo prinesie hmatateľné ekonomické úspory (nižšie náklady na chemikálie, vyššia produkcia) a environmentálne výhody (menšie znečistenie pôdy a vody, príspevok k Zelenému paktu EÚ).

\subsection*{Pre slovenských farmárov}
Rýchlejšia detekcia nákaz a chorôb môže ušetriť až \euro 150-200/ha na nákladoch na ošetrenie.

\subsection*{Pre trh poľnohospodárských technológii}
Vytvára prenosnú metodiku pre ďalšie regióny EÚ a podporuje vznikajúce služby pre klimaticky inteligentné poľnohospodárstvo.

\subsection*{Spoločenské prínosy}
Zlepšená potravinová bezpečnosť, kvalitnejšie produkty, zníženie emisií skleníkových plynov z nadmernej hnojenia a nové kvalifikované pracovné miesta v oblasti prevádzky dronov a AI dátových služieb.

\subsection*{Maximalizácia výsledkov a komunikácia}
\begin{itemize}
    \item Verejné dni poľa s poľnohospodárskymi organizáciami každú sezónu;
    \item Open-access publikácia hlavnej metodiky AI po kontrole IP;
    \item Workshopy s Ministerstvom pôdohospodárstva a regionálnymi družstvami pre usmernenia pri implementácii;
\end{itemize}


\section{Implementácia}

\subsection*{Pracovný plán a ciele (36 mesiacov):}
\begin{description}
    \item[P1 (M1-M6):] Prieskum trhu a dizajn systému $\to$ Špecifikácia pre drony + AI pipeline.
    \item[P2 (M4-M14):] Zber dát (multispektrálne lety, sieť senzorov pôdy) $\to$ Usporiadaný označený dataset $\geq 50$k snímok.
    \item[P3 (M8-M18):] Vývoj a tréning modelov $\to$ Základný AI model (mAP $\geq 0.80$) pre detekciu škodcov/chorôb.
    \item[P4 (M15-M26):] Integrácia hardvéru a nasadenie edge-AI $\to$ Prototyp UAV-AI platformy TRL 6.
    \item[P5 (M24-M34):] Pilotné skúšky na 200 ha farmách $\to$ Overený výkon ($\geq 20$\% zisk výnosu oproti kontrolám).
    \item[P6 (M30-M36):] Finálna optimalizácia, používateľské príručky a balík pre prenos technológií$\to$ Softvérový prototyp + smernice..
\end{description}

\subsection*{Riadenie projektu}
Mesačné hodnotenie pokroku; monitorovanie rizík pomocou cloudového PM nástroja.

\subsection*{Riziká a zmiernenie}
\begin{itemize}
    \item Zlé počasie obmedzujúce lety dronov $\to$ nárazová doba a alternatívne testovacie zariadenia v interiéri.
    \item Nedostatočné označené dáta $\to$ semi-syntetická augmentácia dát a transfer-learning.
    \item Porucha hardvéru $\to$ redundantná jednotka drona + servisná zmluva.
\end{itemize}

\subsection*{Adekvátnosť rozpočtu}
 50 \% Mzdové náklady a ostatné osobné náklady, 15\%  Materiály (drony, GPU server, sensory),  5\% Priame náklady, 5\% Energie a popularizácia. 10 \% Nepriame náklady a služby, 10\% Zdravotné a sociálne poistenie , 5\% Cestovné náklady


%ENGISH
\newpage
\section*{VV-F – MATERIAL INTENT OF THE PROJECT}

\begin {center}
\textbf{\textcolor{blue}{Obligatory Scheme for Applied Research and Development}}
\end {center}

\begin{tabular}{|p{3cm}|p{10cm}|}
\hline       
\\Project title & AGRI-DRONE AI \\ \hline
\\Principal Investigator  & Martin Sabo \\ \hline
\\Applicant organisation & Rubchev Dmytro \\ \hline
\\Statutory representative(s) of the applicant & Ján Lang \\ \hline
\bottomrule
\end{tabular}


\section{Excellence}

The project “Development of an Intelligent Drone System for Agricultural Monitoring with AI Analysis (AGRI-DRONE-AI)” aims to create an integrated platform that combines multispectral drone imaging, edge-AI processing and satellite-weather data fusion to improve precision agriculture in the Slovak Republic.

\subsection*{The originality lies in:}
\begin{enumerate}[label=(\alph*)]
    \item Real-time on-board AI inference for pest, disease and nutrient-stress detection directly in the drone’s edge-computing unit;
    \item Predictive analytics that merge drone imagery with Copernicus EU satellite data and local weather-station feeds;
    \item Actionable agronomic dashboards delivering field-level recommendations to farmers within minutes of a flight.
\end{enumerate}

Current agricultural drone solutions (e.g., DJI Agriculture, PrecisionHawk) mostly supply raw imagery that requires offline processing; our concept shortens the decision cycle and reduces dependency on external IT infrastructure.

\subsection*{State-of-the-art}
EU agriculture-drone market is forecast to reach \euro 7.46 billion by 2025. Research shows that timely pest-detection can reduce yield loss by 15-25 \% and cut pesticide use by up to 30 \%. The team builds on prior national pilot studies in multispectral monitoring of wheat and maize (2023-24) and an internal prototype for CNN-based leaf-disease recognition (accuracy $\approx 92\,\%$)
.
\subsection*{Project objectives:}
\begin{itemize}
    \item Design drone-sensor architecture and edge-AI pipeline for crop-health assessment.
    \item Develop \& train computer-vision models for pest/disease and irrigation-stress recognition.
    \item Integrate predictive models with weather/soil-sensor data for yield-improvement guidance.
    \item Field-validate the prototype on at least 200 ha of pilot farms in western and eastern Slovakia.
    \item Deliver a tested software prototype, AI integration methodology and best-practice adoption guidelines.
\end{itemize}

\subsection*{Feasibility}
Availability of commercial drones and GPUs make the objectives realistic within 36 months.

\subsection*{Principal Investigator’s key outputs (last 5 years – illustrative):}
\begin{itemize}
    \item No
\end{itemize}

\section{Impact}

The system is expected to raise crop yield by 15-25 \% and reduce pesticide consumption by 20-30 \%, bringing tangible economic savings (lower chemical cost, higher production) and environmental benefits (less soil and water contamination, contribution to EU Green Deal).

\subsection*{For Slovak farmers}
Faster detection of pest/disease outbreaks can save up to \euro 150-200/ha in treatment costs.

\subsection*{For the agri-tech market}
Creates a transferable methodology for other EU regions and supports emerging climate-smart-farming services.

\subsection*{Societal benefits}
Improved food-security, higher-quality products, reduction of greenhouse-gas emissions from over-fertilisation.

\subsection*{Maximising results \& communication}
\begin{itemize}
    \item Public field-days with farmers’ associations each season;
    \item Open-access publication of core AI methodology after IP review;
    \item Workshops with Ministry of Agriculture and regional co-ops for adoption guidelines;
\end{itemize}

\section{Implementation}

\subsection*{Work plan \& milestones (36 months):}
\begin{description}
    \item[WP1 (M1-M6):] Market survey \& system design $\to$ Spec-sheet for drones + AI pipeline.
    \item[WP2 (M4-M14):] Data collection (multispectral flights, soil-sensor network) $\to$ Curated labelled dataset $\geq 50$k images.
    \item[WP3 (M8-M18):] Model development \& training $\to$ Baseline AI model (mAP $\geq 0.80$) for pest/disease detection.
    \item[WP4 (M15-M26):] Hardware integration \& edge-AI deployment $\to$ Prototype UAV-AI platform TRL 6.
    \item[WP5 (M24-M34):] Pilot-field trials on 200 ha farms $\to$ Validated performance ($\geq 20\,\%$ yield-gain vs. control).
    \item[WP6 (M30-M36):] Final optimization, user-manuals \& technology-transfer package $\to$ Software prototype + guidelines.
\end{description}

\subsection*{Project management}
Monthly progress reviews; risk-log monitored in cloud-based PM tool.

\subsection*{Risks \& mitigation}
\begin{itemize}
    \item Bad weather limiting drone flights $\to$ buffer period \& alternative indoor test-beds.
    \item Insufficient labelled data $\to$ semi-synthetic data-augmentation \& transfer-learning.
    \item Hardware failure $\to$ redundant drone unit + maintenance contract.
\end{itemize}

\subsection*{Budget adequacy}
 50 \% Wage and other personal costs, 15\% equipment (drones, GPU server, sensors),  5\% Direct costs, 5\% Energy expenses and popularization. 10 \% Indirect costs and services, 10\% Social and health insurance , 5\% travel costs. 

