\section*{VV-A1: Basic information on the project}
    
\subsection*{VV-A1-01:Evidenčné číslo projektu / Project ID }
APVV-24-0000
\subsection*{VV-A1-02:Dátum podania / Date of submission }
11.10.2025
\subsection*{VV-A1-03:Názov projektu / Project title in English }
    SK:Vývoj inteligentného systému dronov na monitorovanie poľnohospodárstva s analýzou AI \\
    EN:Development of an Intelligent Drone System for Agricultural Monitoring with AI Analysis
\subsection*{VV-A1-04:Akronym projektu / Acronym of the project }
    AGRI-DRONE-AI
\subsection*{VV-A1-05:Odbor vedy a techniky / R\&D specialization}

\subsection*{VV-A1-06:Charakter výskumu / R\&D characterization }
\subsection*{VV-A1-07:Začiatok riešenia projektu / Project start }
01.01.2026
\subsection*{VV-A1-08:Koniec riešenia projektu / Project end }
31.12.2028
\subsection*{VV-A1-09:Anotácia (max. 2 000 znakov) / Annotation (max. 2 000 characters) }
Slovak:\\Projekt sa zameriava na vývoj systému na báze dronov pre monitorovanie poľnohospodárstva s využitím umelej inteligencie. Systém využíva drony na zber leteckých údajov o plodinách, analyzuje ich v reálnom čase pomocou algoritmov umelej inteligencie s cieľom zistiť škodcov, choroby, nedostatok živín a potreby zavlažovania a poskytuje farmárom praktické odporúčania. Integruje údaje zo zdrojov, ako sú satelitné snímky (napr. program EÚ Copernicus) a miestnych meteorologických staníc (napr. Slovenský hydrometeorologický ústav), čím optimalizuje presné poľnohospodárstvo v Slovenskej republike. Cieľom je zvýšiť výnosy plodín o 15 – 25 \%, znížiť používanie pesticídov a podporovať udržateľné poľnohospodárske postupy.

V prvej fáze sa vykoná analýza existujúcich technológií dronov (napr. drony DJI Agriculture, PrecisionHawk) a rámcov umelej inteligencie. Druhá fáza zahŕňa vývoj modelov umelej inteligencie pre rozpoznávanie obrazu a prediktívnu analýzu. Tretia fáza testuje prototyp v reálnych podmienkach a meria zlepšenie efektívnosti (očakávané zvýšenie výnosov o 20 \%). Výstupy: Prototyp softvéru, metodika integrácie umelej inteligencie a usmernenia pre zavedenie v poľnohospodárskom sektore.
Projekt sa zaoberá rastúcim trhom s dronmi v poľnohospodárstve v EÚ (podľa trhových správ sa v roku 2025 predpokladá rast na 7,46 miliardy EUR), ako aj klimatické výzvy. Originalita: Kombinácia multispektrálneho zobrazovania pomocou dronov s prediktívnym modelovaním založeným na umelej inteligencii a edge computingom. Prínosy: Ekonomické úspory pre poľnohospodárov, ochrana životného prostredia prostredníctvom zníženého používania chemikálií a podpora cieľov Zelená dohoda pre Európu.
\\

English:\\The project focuses on the development of a drone-based system for agricultural monitoring using artificial intelligence. The system deploys drones to collect aerial data on crops, analyzes it in real-time using AI algorithms to detect pests, diseases, nutrient deficiencies, and irrigation needs, and provides actionable recommendations to farmers. It integrates data from sources like satellite imagery (e.g., Copernicus EU program) and local weather stations (e.g., Slovak Hydrometeorological Institute), optimizing for precision agriculture in the Slovak Republic. The goal is to increase crop yields by 15-25\%, reduce pesticide use, and promote sustainable farming practices.

In the first phase, an analysis of existing drone technologies (e.g., DJI Agriculture drones, PrecisionHawk) and AI frameworks will be conducted. The second phase involves developing AI models for image recognition and predictive analytics. The third phase tests the prototype in real fields, measuring improvements in efficiency (expected 20\% yield increase). Outputs: Software prototype, AI integration methodology, and guidelines for agricultural sector adoption.

The project addresses the growing drone market in agriculture in the EU (projected growth to €7.46 billion in 2025 according to market reports) and climate challenges. Originality: Combination of multi-spectral drone imaging with AI-driven predictive modeling and edge computing. Benefits: Economic savings for farmers, environmental protection through reduced chemical use, and support for EU Green Deal objectives.
\subsection*{VV-A1-10:Žiadateľská organizácia / Co-ordinating organization }
Slovak University of Technology in Bratislava
\subsection*{VV-A1-11:Požadované finančné prostriedky z APVV (v EUR) / Required budget from the agency (in EUR) }
200 000 EUR
\subsection*{VV-A1-12:Spolufinancovanie projektu (v EUR) / Financing from other sources (in EUR)}
0 EUR
\subsection*{VV-A1-13:Celkové náklady na projekt (v EUR) / Total project budget (in EUR) }
200 000 EUR
