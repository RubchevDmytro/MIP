
\documentclass[11pt]{article}
% Define un tipo de columna para la tabla
\newcolumntype{L}[1]{>{\raggedright\arraybackslash}p{#1}}

%%%%%%%%%%%%%%%%%%%%
% INICIO DEL DOCUMENTO
%%%%%%%%%%%%%%%%%%%%
\begin{document}

% --- Page 17 – List of Devices in Project ---
% [Coincide con la pág. 17 del PDF: Zoznam prístrojov]
\section*{Page 17 – List of Devices in Project}

\begin{longtable}{@{} L{0.4\textwidth} L{0.55\textwidth} @{}}
\toprule
\textbf{Name} & \textbf{Description} \\
\midrule
\endfirsthead
% Encabezado para páginas siguientes
\toprule
\textbf{Name} & \textbf{Description} \\
\midrule
\endhead
\bottomrule
\endfoot

Multispectral Agricultural Drone (DJI Agras T50 or equivalent) & Heavy-duty drone equipped with multispectral and RGB cameras, LiDAR altimeter and RTK GPS; used for aerial mapping of crop fields. \\
\addlinespace
AI-Edge Computing Unit & On-board Nvidia Jetson-based processor for real-time image recognition, pest/disease detection and data compression. \\
\addlinespace
Portable Weather \& Soil Station & IoT station measuring soil moisture, temperature, humidity and wind to correlate with drone data. \\
\addlinespace
High-Performance GPU Server & Central workstation (dual GPU) for AI-model training, data fusion with satellite imagery and historical datasets. \\
\addlinespace
Field-testing Kit (markers \& calibration targets) & Set of ground-control points and spectral calibration panels for accurate georeferencing of aerial images. \\
\addlinespace
Data-Storage NAS System & Redundant network-attached storage ($\geq 100$ TB) for archiving flight imagery and AI-processed datasets. \\
\bottomrule
\end{longtable}

% --- Pages 18-19 – Declarations of Honour ---
% [Coincide con las págs. 18-19 del PDF: Čestné vyhlásenie]
\section*{Pages 18-19 – Declarations of Honour}

\textit{These pages normally contain signatures; keep the legal text of the APVV form unchanged and just insert the project / organization names when we generate the final PDF.}

\begin{description}
    \item[Applicant organisation:] Slovak Institute of Smart Agriculture (example – we replace with real applicant)
    \item[Co-operating organisation:] Technical University of Košice – Dept. of AI \& Robotics ( we example – replace as needed)
    \item[Project acronym:] AGRI-DRONE-AI
\end{description}

\textit{Ensure the authorised representatives fill in name, title, place, date and signature in the official form.}

\newpage

% --- Pages 20-23 – VV-F – MATERIAL INTENT OF THE PROJECT ---
% [Coincide con las págs. 20-23 del PDF: VV-F - VECNÝ ZÁMER PROJEKTU]
\section*{Page 20 - Excellence #1}

\section{Excellence}

The project “Development of an Intelligent Drone System for Agricultural Monitoring with AI Analysis (AGRI-DRONE-AI)” aims to create an integrated platform that combines multispectral drone imaging, edge-AI processing and satellite-weather data fusion to improve precision agriculture in the Slovak Republic.

\subsection*{The originality lies in:}
\begin{enumerate}[label=(\alph*)]
    \item Real-time on-board AI inference for pest, disease and nutrient-stress detection directly in the drone’s edge-computing unit;
    \item Predictive analytics that merge drone imagery with Copernicus EU satellite data and local weather-station feeds;
    \item Actionable agronomic dashboards delivering field-level recommendations to farmers within minutes of a flight.
\end{enumerate}

Current agricultural drone solutions (e.g., DJI Agriculture, PrecisionHawk) mostly supply raw imagery that requires offline processing; our concept shortens the decision cycle and reduces dependency on external IT infrastructure.

\subsection*{State-of-the-art}
EU agriculture-drone market is forecast to reach \euro 7.46 billion by 2025. Research shows that timely pest-detection can reduce yield loss by 15-25 \% and cut pesticide use by up to 30 \%. The team builds on prior national pilot studies in multispectral monitoring of wheat and maize (2023-24) and an internal prototype for CNN-based leaf-disease recognition (accuracy $\approx 92\,\%$)
.
\subsection*{Project objectives:}
\begin{itemize}
    \item Design drone-sensor architecture and edge-AI pipeline for crop-health assessment.
    \item Develop \& train computer-vision models for pest/disease and irrigation-stress recognition.
    \item Integrate predictive models with weather/soil-sensor data for yield-improvement guidance.
    \item Field-validate the prototype on at least 200 ha of pilot farms in western and eastern Slovakia.
    \item Deliver a tested software prototype, AI integration methodology and best-practice adoption guidelines.
\end{itemize}

\subsection*{Feasibility}
Availability of commercial drones and GPUs plus our lab’s prior datasets make the objectives realistic within 36 months.

\subsection*{Principal Investigator’s key outputs (last 5 years – illustrative):}
\begin{itemize}
    \item Prototype CNN for wheat-rust detection (TRL 5 $\to$ yield loss $\downarrow 18\,\%$)
.
    \item Co-author of 2 EU-Horizon papers on precision-agriculture AI.
    \item Open-source soil-moisture fusion algorithm (adopted by 3 regional start-ups).
\end{itemize}

\subsection*{Competence of partners:}
\begin{itemize}
    \item Applicant organisation brings agronomy test-fields and UAV-pilot team.
    \item Co-operating university contributes expertise in deep-learning and remote-sensing.
    \item Industrial partner (AgroTech Solutions s.r.o.) will advise on market transfer.
\end{itemize}

\subsection*{Involvement of young researchers:}
4 PhD students (AI \& Robotics, age $\leq 30$) plus 2 MSc students in geoinformatics will join data-labelling, field-validation and dissemination tasks.
