\documentclass[a4paper,12pt]{article}
\usepackage[utf8]{inputenc}
\usepackage[T1]{fontenc}
\usepackage[slovak]{babel}
\usepackage{lmodern}
\usepackage{url}
\usepackage{geometry}
\geometry{margin=2cm}
\usepackage{bibentry}
\begin{document}

\title{Zameranie projektu}
\nobibliography{zameranie}
\author{Stanislav Sasyn}
\date{\today}
\maketitle

\section*{VV-A1-03: Názov projektu / Project title in English}
Názov projektu: V \\
Project title in English: 
\section*{VV-A1-04: Akronym projektu / Acronym of the project}
Akronym: SMARTSHOP-AI

\section*{VV-A1-09: Anotácia / Annotation }
Projekt sa zameriava na vývoj  systému pre optimalizáciu nákupov potravín pomocou umelej inteligencie. 
Systém analyzuje zoznam produktov používateľa, lokalizuje najbližšie obchody (napr. Tesco, Lidl v Slovenskej republike), 
porovnáva ceny z reálnych zdrojov dát (napr. Numbeo\footnote{\bibentry{numbeo}} alebo oficiálny cenový trackér Ministerstva pôdohospodárstva SR\footnote{\bibentry{mp}}) 
a navrhuje najlepšie varianty nákupu s ohľadom na cenu, vzdialenosť a ekológiu. Cieľom je ušetriť čas a peniaze spotrebiteľom, 
znížiť potravinový odpad a podporiť udržateľné nákupy.

V prvej etape sa vykoná analýza existujúcich aplikácií (napr. Grocery Dealz, WiseList) a integrácia zdrojov dát. 
Druhá etapa zahŕňa vývoj ML algoritmov pre predikciu cien a optimalizáciu trás. 
Tretia etapa testuje prototyp v reálnom prostredí s meraním úspor (očakávané 10-20\% na nákladoch). 
Výstupy: Softvérový prototyp, metodika integrácie API a odporúčania pre retail sektor.

Projekt reaguje na rastúce ceny potravín v EÚ (podľa Eurostatu +2.7\% v 2025\footnote{\bibentry{eurostat}}) a digitalizáciu. 
Originalita: Kombinácia geolokácie s ekológiou a personalizáciou. 
Prínosy: Ekonomické úspory pre domácnosti, podpora lokálnych obchodov a znižovanie emisií CO2.

\bibliographystyle{plain}
\end{document}
